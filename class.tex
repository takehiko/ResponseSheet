%#!platex --kanji=utf8 class
\documentclass[12pt,a4paper]{jsarticle}

\pagestyle{empty}
\unitlength=1mm
\def\sheettitle{情報セキュリティ 授業メモ}
%\def\sheettitle{情報処理II 小テスト解答用紙}
\def\studentid{学生番号}
%\def\studentid{学籍番号}
\def\cutofflinetype{2} % 0:境界線なし,1:端にわずかの境界線,2:境界線を引く
\def\drawcutoffline{%
 \ifnum\cutofflinetype=1
 \put(-85,0){\line(1,0){2}}\put(85,0){\line(-1,0){2}}
 \else\ifnum\cutofflinetype=2
 \put(-85,0){\line(1,0){170}}
 \fi\fi}

\begin{document}

\newsavebox\toan
\sbox{\toan}{\hbox{%
 \begin{picture}(0,62)
  \thinlines
  \drawcutoffline
  \put(0,53){\makebox(0,0){{\huge\textbf{\sheettitle}}}}
  \put(-81,43){\makebox(0,0)[l]{{\large\textbf{\studentid:}}}}
  \put(-19,43){\makebox(0,0)[l]{{\large\textbf{氏名:}}}}
  \put(81,43){\makebox(0,0)[r]{{\large\textbf{実施日: \hspace{8mm}月 \hspace{8mm}日}}}}
  \thicklines
  \put(-82,40){\line(1,0){60}}
  \put(-20,40){\line(1,0){57}}
  \put(81,40){\line(-1,0){42}}
  \thinlines
  \put(-62,40){\framebox(40,8){}}
  \multiput(-57,40)(5,0){7}{\line(0,1){8}}
 \end{picture}
}}

\begin{center}
 \begin{picture}(0,0)
  \put(0,8){\makebox(0,0){\usebox{\toan}}}
  \put(0,-54){\makebox(0,0){\usebox{\toan}}}
  \put(0,-116){\makebox(0,0){\usebox{\toan}}}
  \put(0,-178){\makebox(0,0){\usebox{\toan}}}
 \end{picture}
\end{center}

\end{document}
