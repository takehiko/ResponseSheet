%#!platex --kanji=utf8 seminar
\documentclass[12pt,a4paper]{jsarticle}

\pagestyle{empty}
\unitlength=1mm
\newif\ifprintflipside
\printflipsidetrue % 裏面印刷する
%\printflipsidefalse % 裏面印刷しない
\def\cutofflinetype{1} % 0:境界線なし,1:端にわずかの境界線,2:境界線を引く
\def\drawcutoffline{%
 \ifnum\cutofflinetype=1
 \put(-85,0){\line(1,0){2}}\put(85,0){\line(-1,0){2}}
 \else\ifnum\cutofflinetype=2
 \put(-85,0){\line(1,0){170}}
 \fi\fi}

\begin{document}

\newsavebox\toan
\sbox{\toan}{\hbox{%
 \begin{picture}(0,62)
  \thinlines
  \drawcutoffline
  \put(0,53){\makebox(0,0){{\huge\textbf{メモ用紙}}}}
  \put(-81,43){\makebox(0,0)[l]{{\large\textbf{学年・氏名:}}}}
  \put(80,43){\makebox(0,0)[r]{{\large\textbf{日付: \hspace{15mm}年 \hspace{10mm}月 \hspace{10mm}日}}}}
  \thicklines
  \put(-82,40){\line(1,0){95}}
  \put(81,40){\line(-1,0){64}}
 \end{picture}
}}

\newsavebox\uraA
\sbox{\uraA}{\hbox{%
 \begin{picture}(0,62)
  \thinlines
  \drawcutoffline
  \put(-81,6){\makebox(0,0)[l]{この用紙を受け取った人は,主に\underline{研究内容}に関するメモを書くよう努めてください.}}
 \end{picture}
}}

\newsavebox\uraB
\sbox{\uraB}{\hbox{%
 \begin{picture}(0,62)
  \thinlines
  \drawcutoffline
  \put(-81,6){\makebox(0,0)[l]{この用紙を受け取った人は,主に\underline{発表方法}(話し方など)に関するメモを書くよう努めてください.}}
 \end{picture}
}}

\newsavebox\uraC
\sbox{\uraC}{\hbox{%
 \begin{picture}(0,62)
  \thinlines
  \drawcutoffline
  \put(-81,6){\makebox(0,0)[l]{この用紙を受け取った人は,最低1回質問をし,その\underline{質問と回答}を書いてください.}}
 \end{picture}
}}

\begin{center}
 \begin{picture}(0,0)
  \put(0,8){\makebox(0,0){\usebox{\toan}}}
  \put(0,-54){\makebox(0,0){\usebox{\toan}}}
  \put(0,-116){\makebox(0,0){\usebox{\toan}}}
  \put(0,-178){\makebox(0,0){\usebox{\toan}}}
 \end{picture}
\end{center}
\ifprintflipside
\newpage
\begin{center}
 \begin{picture}(0,0)
  \put(0,8){\makebox(0,0){\usebox{\uraA}}}
  \put(0,-54){\makebox(0,0){\usebox{\uraB}}}
  \put(0,-116){\makebox(0,0){\usebox{\uraC}}}
  \put(0,-178){\makebox(0,0){\usebox{\uraA}}}
 \end{picture}
\end{center}
\fi

\end{document}
